
\documentclass[submit]{harvardml}

% You don't need to change these.
\course{CS181-S19}
\assignment{Assignment \#3}
\duedate{11:59pm March 22, 2019}

\usepackage[OT1]{fontenc}
\usepackage[colorlinks,citecolor=blue,urlcolor=blue]{hyperref}
\usepackage[pdftex]{graphicx}
\usepackage{subfig}
\usepackage{fullpage}
\usepackage{amsmath}
\usepackage{amssymb}
\usepackage{color}
\usepackage{todonotes}
\usepackage{listings}
\usepackage{common}
\usepackage{bm}

\usepackage[mmddyyyy,hhmmss]{datetime}

\definecolor{verbgray}{gray}{0.9}

\lstnewenvironment{csv}{%
  \lstset{backgroundcolor=\color{verbgray},
  frame=single,
  framerule=0pt,
  basicstyle=\ttfamily,
  columns=fullflexible}}{}

\begin{document}
\begin{center}
{\Large Homework 3: Max-Margin, Ethics, Clustering}\\
\end{center}
\subsection*{Introduction}

This homework assignment will have you work with max-margin methods
and clustering, as well as an ethics assignment.  The aim of the
assignment is (1) to further develop your geometrical intuition behind
margin-based classification and decision boundaries, (2) try coding a
simple K-means classifier, and (3) to have you reflect on the ethics
lecture and to address the scenario discussed in class in more depth
by considering the labor market dynamically.

We encourage you to first read the Bishop textbook coverage of these topics,
particularly: Section 7.1 (Max-Margin and SVMs) and Section 9.1 (Clustering).
Chapters 5 and 6 of the student textbook are also relevant.

There is a mathematical component and a programming component to this
homework.  Please submit your PDF, tex, and Python files to Canvas,
and push all of your work to your GitHub repository. If a question
requires you to make any plots, like Problem 2, please include those
in the writeup.

\newpage
%%%%%%%%%%%%%%%%%%%%%%%%%%%%%%%%%%%%%%%%%%%%%
% Problem 1
%%%%%%%%%%%%%%%%%%%%%%%%%%%%%%%%%%%%%%%%%%%%%
\begin{problem}[Fitting an SVM by hand, 7pts]
  For this problem you will solve an SVM without the help of a
  computer, relying instead on principled rules and properties of
  these classifiers.

Consider a dataset with the following 7 data points each with $x \in \reals$ : \[\{(x_i, y_i)\}_i =\{(-3 , +1) , (-2 , +1 ) , (-1,  -1 ), (0, +1), ( 1 , -1 ), ( 2 , +1 ) , (-3 , +1 )\}\] Consider
mapping these points to $2$ dimensions using the feature vector $\bphi(x) =  (x, -\frac{8}{3}x^2 + \frac{2}{3}x^4 )$. The hard margin classifier training problem is:
%
\begin{align*}
  &\min_{\mathbf{w}, w_0} \|\mathbf{w}\|_2^2 \label{eq:dcp} \\
  \quad \text{s.t.} \quad & y_i(\mathbf{w}^\top \bphi(x_i) + w_0) \geq 1,~\forall i \in \{1,\ldots, n\}\notag
\end{align*}

The exercise has been broken down into a series of questions, each
providing a part of the solution. Make sure to follow the logical structure of
the exercise when composing your answer and to justify each step.

\begin{enumerate}
\item Plot the training data in $\reals^2$ and draw the decision boundary
of the max margin classifer.
%
\item  What is the value of the margin achieved by the optimal
decision boundary? 
%
\item What is a vector that is orthogonal to the decision boundary?

%
\item Considering discriminant $h(\bphi(x);\boldw,w_0)=\boldw^\top\bphi(x) +w_0$, 
give an expression for {\em all possible} $(\boldw,w_0)$ that define
the optimal decision boundary. Justify your answer.

  \item Consider now the training problem. Using your answers so far,
    what particular solution to $\boldw$ will be optimal for this
    optimization problem?

  \item Now solve for the corresponding value of $w_0$, using your
    general expression from part~(4.) for the optimal decision
    boundary.  Write down the discriminant function
    $h(\bphi(x);\boldw,w_0)$.


\item What are the support vectors of the classifier?  Confirm that
  the solution in part~(6.) makes the constraints above binding for
  support vectors.

\end{enumerate}

\end{problem}
\subsection*{Solution}

\newpage
%%%%%%%%%%%%%%%%%%%%%%%%%%%%%%%%%%%%%%%%%%%%%
% Problem 2
%%%%%%%%%%%%%%%%%%%%%%%%%%%%%%%%%%%%%%%%%%%%%

\begin{problem}[K-Means, 10pts]

For this problem you will implement K-Means clustering from
scratch. Using \texttt{numpy} is fine, but don't use a third-party
machine learning implementation like \texttt{scikit-learn}. You will
then apply this approach to clustering of image data.

We have provided you with the MNIST dataset, a collection of
handwritten digits used as a benchmark of image recogntion (you can
learn more about the data set at
\url{http://yann.lecun.com/exdb/mnist/}). The MNIST task is widely
used in supervised learning, and modern algorithms with neural
networks do very well on this task.

Here we will apply unsupervised learning to MNIST. You have been given
representations of 6000 MNIST images, each of which are $28\times28$
greyscale handwritten digits. Your job is to implement K-means
clustering on MNIST, and to test whether this relatively simple
algorithm can cluster similar-looking images together.

The given code loads the images into your environment as a 6000x28x28
array.  In your code, you may use the $\ell_2$ norm as your distance
metric. (You should feel free to explore other metrics than the
$\ell_2$ norm, but this is strictly optional.)

\begin{itemize}

\item Starting at a random initialization and some choice of K, plot
  the K-means objective function as a function of iteration and verify
  that it never increases.

\item Run the K-means algorithm from several different restarts for
  different values of K.  Plot the final K-means objective as a
  function of K with errorbars over the random restarts.  How does the
  objective and the variance of the objective change with K?  
  
\item For $K=10$, for a couple of random restarts, show the mean
  images for each cluster.  To render an image, use the pyplot
  \texttt{imshow} function.

\item Now, before running K-means, standardize the data.  That is,
  center the data first so that each pixel has mean 0 and variance 1
  (except for any pixels that have zero variance).  For $K=10$, for a
  couple of random restarts, show the mean images for each cluster.
  Compare them to the previous part.

\end{itemize}

As in past problem sets, please include your plots in this
document. (There may be several plots for this problem, so feel free
to take up multiple pages.)

\end{problem}

\subsection*{Solution}


\newpage
%%%%%%%%%%%%%%%%%%%%%%%%%%%%%%%%%%%%%%%%%%%%%
% Problem 3
%%%%%%%%%%%%%%%%%%%%%%%%%%%%%%%%%%%%%%%%%%%%%
\begin{problem}[Ethics Assignment, 10pts]
\textbf{Our class activity:}

\begin{quote}
Hiring at Forever 28. Forever 28 has hired a new computer science team
to design an algorithm to predict the success of various job
applicants to sales positions at Forever 28. As you go through the
data and design the algorithm, you notice that African-American sales
representatives have significantly fewer average sales than white
sales representatives, and the most profit comes from interactions
between white female sales representatives and white male customers.
The algorithm's output recommends hiring far fewer African-Americans
than white applicants, even after the percentage of applications from
people of various races are adjusted for, and having those sales
representatives target white male customers.
\end{quote}

In class, we assumed that the problem was \textit{static}: given
historical data, such as data about sales performance, who should
Forever 28 hire right now?  In this follow-up assignment, think
about consumer behavior and firm hiring practice dynamically. Looking
at features of the labor market dynamically allows you different
degrees of freedom in your model. For example, in class, you probably
took consumer preference about the race of their sales representative
as given. What would happen if you assumed that consumer preference
could vary over time (say, on the basis of changing racial
demographics in the sales force)?

\textbf{Your new case:}
\begin{quote}
The US Secretary of Labor has heard about your team's success with Forever 28 and comes to you with a request. The Department of Labor wants to reduce disparate impact discrimination in hiring. They want you to come up with a model of fair hiring practices in the labor market that will reduce disparate impact while also producing good outcomes for companies. 
\end{quote}

Write two to three paragraphs that address the following:
\begin{itemize}
\item What is disparate impact, and how does it differ from disparate treatment?  
\item What are the relevant outcomes, for both workers and companies?  Are these outcomes measurable?  
\item What are some properties of your algorithm that might produce those socially good results?  Think about constraints that you might build in, such as the fairness constraints that we discussed in class, or how you might specify the prediction task that we are asking the machine to optimize.  [Optional] What trade-offs might your algorithm have to balance?  
\item Recommend a deployment strategy.  Are there any features of the data collection, algorithm implementation, or broader context that make you wary?  Describe how your algorithm might fit into the broader context of a company (e.g. hiring, training, marketing, sales).  
\end{itemize} 

We expect clear, concise, and thoughtful engagement with this question, which includes providing your reasoning for your answers.  In your response, depth is more important than breadth.  For example, in question 1, you could simply choose profit for the outcome that companies may be interested in; you can run with a single fairness criterion. We do \emph{not} expect you to do any outside research, though we encourage you to connect to lecture materials where relevant.

\end{problem}

\subsection*{Solution}
  

\newpage

\begin{itemize}
    \item Name:
    \item Email:
    \item Collaborators:
    \item Approximately how long did this homework take you to complete (in hours):   
\end{itemize}

\end{document}


















































